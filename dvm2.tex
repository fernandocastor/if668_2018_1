\documentclass[a4paper]{article}

%% Language and font encodings
\usepackage[brazil]{babel}
\usepackage[utf8x]{inputenc}
\usepackage[T1]{fontenc}

%% Sets page size and margins
\usepackage[a4paper,top=3cm,bottom=2cm,left=3cm,right=3cm,marginparwidth=1.75cm]{geometry}

%% Useful packages
\usepackage{amsmath}
\usepackage{graphicx}
\usepackage[colorinlistoftodos]{todonotes}
\usepackage[colorlinks=true, allcolors=blue]{hyperref}

\title{IF689 - Informática Teórica}
\author{Davi Vicente Magnata}

\begin{document}
\maketitle



\section{Introducão}

\hspace{\parindent} A cadeira de informação teórica é \textbf{obrigatória} e forma a ementa para tanto o curso de Ciência da computação como também o curso de Engenharia da computação, ela é dada pelos professores Ruy Queiroz (CC) e Paulo Fonseca (EC).
    Esse curso é divido em três (3) áreas principais: 
\begin{itemize}
\item Autômatos: Lida com as definições e propriedades de modelos matemáticos de computação.


\item Complexidade: "O que faz alguns problemas computacionalmente difíceis e outros fáceis?"

\item Computabilidade: "Como separar os problemas que possuem solução computacional daqueles que não possuem?"

No curso são estudados três (3) modelos de computação :Autômatos finitos, Autômatos com pilha e Máquinas de Turing , além de outros conceitos importantes para a área.
\end{itemize}
    
    \begin{figure}[h]
\centering
\includegraphics[width=0.5\textwidth]{Turing.jpg}
\caption{\label{fig:frog}Alan Turing | Licença :Domínio Público}
\end{figure}

    

\section{Relevância}

\hspace{\parindent} Essa cadeira trabalha e desenvolve importantes conhecimentos que ajudarão o aluno a pensar computação de uma maneira completa, abordando vários conceitos que serão importantes ferramentas nessa área para se desenvolver sistemas mais elegantes e eficientes.

\hspace{\parindent} Além disso a resolução de problemas e a habilidade de exprimir-se claramente são conhecimentos desenvolvidos aqui tem um valor duradouro.

\section{Cadeiras Relacionadas}



\begin{table}[h]
\centering
\caption{Cadeiras Relacionadas}
\label{my-label}
\begin{tabular}{l|c|}
\hline
IF774 - COMPLEXIDADE DESCRITIVA                                & Trabalha com a complexidade , uma das grandes áreas da IF689.                                                                     \\ \hline
\multicolumn{1}{|l|}{IF772 - LAMBDA CÁLCULO TEORIA TIPOS}      & Outro modelo de computação, não abordado na cadeira original mas faz parte da área de modelos computacionais, estudados na IF689. \\ \hline
\multicolumn{1}{|l|}{IF778 - SEMINARIO EM INFORMATICA TEORICA} & Seminários sobre temas abordados na área de informação teórica.                                                                   \\ \hline
\multicolumn{1}{|l|}{IF769 - TEORIA DA RECURSÃO}               & Além de trabalhar com temas como computabilidade , essa cadeira também possui máquinas de Turing na sua ementa.                   \\ \hline
IF776 - TOPICOS AVANÇADOS INF. TEÓRICA                         & Expande sobre a base com tópicos avançados envolvendo a informática teórica.                                                      \\ \hline

\end{tabular}
\end{table}

\cite{1}
\cite{2}
\cite{3}
\bibliographystyle{alpha}
\bibliography{dvm2.bib}


\end{document}