\documentclass[a4paper]{article}

%% Language and font encodings
\usepackage[brazil]{babel}
\usepackage[utf8x]{inputenc}
\usepackage[T1]{fontenc}

%% Sets page size and margins
\usepackage[a4paper,top=3cm,bottom=2cm,left=3cm,right=3cm,marginparwidth=1.75cm]{geometry}

%% Useful packages
\usepackage{amsmath}
\usepackage{graphicx}
\usepackage[colorinlistoftodos]{todonotes}
\usepackage[colorlinks=true, allcolors=blue]{hyperref}

\title{IF681 - Interfaces Usuário Máquina}
\author{Gabriel Braz}

\begin{document}
\maketitle

\begin{figure}[h]
\centering
\includegraphics[width=0.8\textwidth]{IUM.png}
\caption{\label{fig:IUM}\cite{fig_IUM}Este diagrama representa a interação entre a máquina e o computador, através da interface.}
\end{figure}

\section{Introdução}

Interfaces Usuário-Máquina\cite{site_disciplina} é uma disciplina obrigatória, ofertada ao curso de ciência da computação no quarto período, lecionada pelo Professor Alex Sandro Gomes. Interfaces Usuário-Máquina estuda a interação entre o usuário e o computador, abordando temas como fundamentos da interação humano-computador, e os fatores humanos\cite{site_cadeira}, buscando desenvolver softwares com um design centrado no usuário. Interfaces com o Usuário é inserida na área de Tecnologia e Sistema de Computação.

\section{Relevância}

A relevância desta disciplina está no fato dela ser a base do contato entre o homem e o computador. Isto é, ela estuda o que as pessoas necessitam e desejam, de forma a melhorar o design de um software, para que, este atenda às necessidades de seus usuários, possuindo uma maior usabilidade, facilitando, desta forma, a interação ( Figura \ref{fig:IUM}), e proporcionando uma melhor experiência. A usabilidade é importante porque pode torna uma ferramenta fácil de usar, ou frustrante, para o desenvolvedor, essa pode ser a diferença entre o fracasso e sucesso.

\subsection{Pontos Positivos}

\begin{itemize}
\item Lecionada no ínicio do curso;
\item Programação voltada para o usuário;
\end{itemize}

\subsection{Pontos Negativos}

\begin{itemize}
\item Pouca carga horária: apenas 40 horas;
\end{itemize}

\section{Relação com outras disciplinas}

\begin{figure}[t]
\centering
\includegraphics[width=0.8\textwidth]{IHC.png}
\caption{\label{fig:IHC}\cite{fig_IUM}Aqui podemos ver diversas áreas que Interfaces Usuário Máquina(IHC) Interage.}
\end{figure}

Por se tratar da comunicação mais básica entre o computador e o usuário, esta disciplina entra em contato com diversas áreas, dentro e fora da computação, como visto na Figura \ref{fig:IHC}.

\begin{table}[h]
\centering
\caption{Disciplina Relacionadas\cite{perfil}}
\label{Table}
\begin{tabular}{|l|l|}
\hline
IF669 - Introdução a Programação & Pré-requisito \\ \hline
\end{tabular}
\end{table}

\bibliographystyle{plain}
\bibliography{gbcs}

\end{document}