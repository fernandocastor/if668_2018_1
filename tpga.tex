\documentclass[a4paper]{article}

%% Language and font encodings
\usepackage[english]{babel}
\usepackage[utf8x]{inputenc}
\usepackage[T1]{fontenc}

%% Sets page size and margins
\usepackage[a4paper,top=3cm,bottom=2cm,left=3cm,right=3cm,marginparwidth=1.75cm]{geometry}

%% Useful packages
\usepackage{amsmath}
\usepackage{graphicx}
\usepackage[colorinlistoftodos]{todonotes}
\usepackage[colorlinks=true, allcolors=blue]{hyperref}

\title{IF683 - Projeto de Desenvolvimento}
\author{Thomas Patrick Gomes de Alcântara.}
\date{07 de Maio de 2018.}

\begin{document}
\maketitle


\section{Introdução}
A disciplina Projeto de Desenvolvimento\cite{wiki}, conhecida também pelo seu famoso apelido "Projetão", é uma disciplina ofertada aos estudantes de Ciência e Engenharia da Computação do CIN e aos estudantes de Design.
De acordo com o manual da Disciplina \cite{manual}, ela serve para os estudantes colocarem em prática o que já foi absorvido em termos de conhecimento durante os primeiros períodos do curso de modo que seja um processo interdisciplinar para amadurecer alguma ideia de projeto futuro e o modo de trabalhar em equipe com pessoas de vários outros cursos.


\section{Relevância.}
Essa disciplina tem grande relevância nos cursos de computação devido à sua natureza interdisciplinar, pois ela marca o final do ciclo básico do curso e o começo do ciclo profissional, então os alunos têm a oportunidade de usar todo o conteúdo adquirido durante o ciclo básico para usá-lo em um projeto real com métodos de desenvolvimento, prazos e responsabilidades usados nos ambientes de trabalho reais de desenvolvimento e depois apresentá-lo no Projetão Demo Day , que é um evento onde os projetos são apresentados ao público no CIN como um evento real de inovação. Por conta desses eventos, alguns projetos apresentados pelos alunos foram visualizados pelas "pessoas certas", que incentivaram o projeto a tomar um rumo profissional e virarem empresas reais. Como exemplo de empresas que avançaram do projeto da disciplina para uma empresa real temos o Coteaqui, um site que cota preços de materiais de construção para construtoras e existe até hoje, provando o sucesso da ideia que começou na disciplina.
 
\subsection{Pontos Positivos}

\begin{itemize}
\item Reúne conceitos de várias disciplinas.
\item Apresenta e prepara os alunos para os métodos e desafios do mercado.
\item Promove o trabalho em equipe entre alunos de diferentes cursos e centros.
\item Permite que os projetos sejam visualizados por empresas e empresários.

\end{itemize}

\subsection{Pontos negativos}
\begin{itemize}
\item O escopo do projeto é definido pelo professor no começo do período, prejudicando alguns alunos que já vêm desenvolvendo a ideia de um projeto desde o começo do curso.
\end{itemize}


\section{Relação com outras disciplinas.}

A disciplina, por ter como principal objetivo levar os alunos a uma abordagem prática e ao desenvolvimento de projetos, pode-se dizer que se relaciona com quase todas as outras disciplinas que o aluno já cursou ou irá cursar, mas as fundamentais para o desenvolvimento do projeto serão listadas na tabela abaixo:

\begin{table}[ht]
\centering
\begin{tabular}{|l|l|}
\hline
\multicolumn{1}{|l|}{} & \multicolumn{1}{l|}{- Usa os fundamentos de programação aprendidos desde o começo do curso.} \\ \cline{2-2}
 & -  Usa os conceitos de estruturas de dados para criar seu app/software. \\ \cline{2-2}
IF672 - Algoritmos e Estrutura da dados. & - Buscar otimizar e tornar seu projeto mais eficiente.\\ \cline{2-2}
 & - Usar o conhecimento em custo de desenvolvimento para organização dos dados. \\ \cline{2-2}
  \cline{1-1}              
 &- Utilizar as noções de projeto de desenvolvimento de software. \\ \cline{2-2}
 & - Saber os requisitos necessários para a execução do projeto.\\ \cline{2-2}
IF682- Engenharia de Software e Sistemas. & -  Saber os conceitos de manutenção de software, gerência e teamwork. \\ \cline{2-2}
 & -  Utilizar as metodologias ágeis de desenvolvimento.\\ \cline{2-2}
  & - Executar o processo de desenvolvimento de modo otimizado na equipe.\\ \cline{2-2}
  \cline{1-1}              
 & - Ter a ideia de quem você quer ajudar com seu projeto. \\ \cline{2-2}
IF679- Informática e sociedade. & - Entender como seu projeto irá impactar na sociedade \\ \cline{2-2}
 & - Saber como será usado seu projeto pelas pessoas.\\ \cline{2-2}
  \cline{2-2}
  \cline{1-1}   
  
\end{tabular}
\end{table}

\newpage
\bibliographystyle{alpha}
\bibliography{tpga}

\end{document}