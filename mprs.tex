\documentclass[a4paper]{article}

%% Language and font encodings
\usepackage[brazil]{babel}
\usepackage[utf8x]{inputenc}
\usepackage[T1]{fontenc}

%% Sets page size and margins
\usepackage[a4paper,top=3cm,bottom=2cm,left=3cm,right=3cm,marginparwidth=1.75cm]{geometry}

%% Useful packages
\usepackage{amsmath}
\usepackage{graphicx}
\usepackage[colorinlistoftodos]{todonotes}
\usepackage[colorlinks=true, allcolors=blue]{hyperref}

\title{IF687 - Introdução à Multimídia}
\author{Magnon Paulino Ramos de Souza}

\begin{document}
\maketitle

\section{Introdução}

A disciplina de Introdução à Multimídia\cite{st} é um componente obrigatório do perfil atual do curso de Ciência da Computação, ministrada no 5º período e tem como pré-requisito Introdução à Programação (IF669).

Os principais tópicos da disciplina são:
\begin{itemize}
\item Fundamentos de Multimídia\cite{ch04};
\item Fundamentos de Realidade Virtual\cite{bu03} (VR); 
\item Fundamentos de Realidade Aumentada (AR);
\item Hardwere para multimídia, AR e VR;
\item Ambientes de desenvolvimento;
\item Aplicações de Multimídia, AR e VR\cite{bow05}.
\end{itemize}

O enfoque da disciplina é operacional, possibilitando o desenvolvimento da capacidade de criar mundos virtuais e componentes multimídia específicos para esses mundos. Com ênfase na Internet, objetiva-se aplicações nas mais diversas áreas, como por exemplo, em educação, visualização de dados, arquitetura e urbanismo, medicina, entretenimento, etc\dots


\section{Relevância}

A disciplina aparece na grade curricular com o principal propósito de introduzir o campo da criação e avaliação de mundos virtuais de forma a otimizar a relação humana com a máquina.

Visto que existem as mais diversas formas de transmitir a informação e o mais amplo leque de conteúdos, a disciplina surge como um instrumental que possibilita a manipulação dessa diversidade para transmitir os dados de forma mais clara.

Essa praticidade é de extrema importância dentro da grade curricular pois possibilita a criação de conteúdo que interajam com o usuário de diversas formas, diminui custos operacionais e facilita o aprendizado e retenção da informação.

Um dos grandes pontos positivos da disciplina é o seu grande potencial. Ela possibilita os mais diversos tipos de projetos e cria conexões para diversos campos da computação, a depender da abordagem dentro dos diferentes tipos de mídia.\newpage


\section{Relação com outras disciplinas}
Como uma disciplina introdutória no campo da multimídia, ela faz conexão com diversas disciplinas do curso, sendo base para elas ou sendo resultado de um aprofundamento delas.
\begin{table}[h]
\centering
\caption{Relação de Introdução à Multimídia com outras disciplinas}
\label{my-label}
\begin{tabular}{|l|l|}
\hline
\multicolumn{1}{|c|}{\textbf{Disciplinas}}                                                    & \multicolumn{1}{c|}{\textbf{Relação}}                                                                                                                           \\ \hline
\textit{IF669 - Introdução à Programação}                                                     & \begin{tabular}[c]{@{}l@{}}Fornece os fundamentos iniciais de interface e interação \\ com o usuário.\end{tabular}                                              \\ \hline
\textit{IF680 - Processamento Gráfico}                                                        & \begin{tabular}[c]{@{}l@{}}Introduz a noção de dados visuais e interface de entrada \\ e saída.\end{tabular}                                                    \\ \hline
\textit{\begin{tabular}[c]{@{}l@{}}IF793 - Projeto Implementação de \\ Jogos 2D\end{tabular}} & \begin{tabular}[c]{@{}l@{}}Para a implementação de jogos, é necessária a compreensão\\ da interação sonora e visual entre jogo e jogador.\end{tabular}         \\ \hline
\textit{IF755 - Realidade Virtual}                                                            & \begin{tabular}[c]{@{}l@{}}Para criação desses espaços virtuais é essencial a \\ compreensão do papel das diversas mídias (visual, \\ áudio, etc).\end{tabular} \\ \hline
\textit{IF681 - Interfaces Usuário-Máquina}                                                   & \begin{tabular}[c]{@{}l@{}}Estabelecer a comunicação do usuário com máquina exige\\ entendimento de como melhor estabelecer essa interação.\end{tabular}        \\ \hline
\end{tabular}
\end{table}

\bibliographystyle{unsrt}
\bibliography{sample}

\end{document}