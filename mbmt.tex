\documentclass[a4paper]{article}

%% Language and font encodings
\usepackage[brazil]{babel}
\usepackage[utf8x]{inputenc}
\usepackage[T1]{fontenc}

%% Sets page size and margins
\usepackage[a4paper,top=3cm,bottom=2cm,left=3cm,right=3cm,marginparwidth=1.75cm]{geometry}

%% Useful packages
\usepackage{amsmath}
\usepackage{graphicx}
\usepackage[colorinlistoftodos]{todonotes}
\usepackage[colorlinks=true, allcolors=blue]{hyperref}

\title{IF685 - Gerenciamento de Dados e Informação}
\author{Matheus Belfort de Moura Torres}

\begin{document}
\maketitle

\section{Introdução}

 Gerenciamento de dados e informação é uma cadeira obrigatória tanto para os alunos de ciência da computação\cite{CinSecGrad-cc} quanto para os de engenharia da computação\cite{CinSecGrad-ec}, a disciplina introduz os sistemas de compartilhamento, armazenamento e gerenciamento de dados. São introduzidos conceitos como e banco de dados, uma coleção de dados, que devem ter uma série de propriedades\cite{elmasri2011sistemas}. Essa área está intimamente ligada a áreas como Big Data, a nova era da exploração e utilização de dados\cite{IBM:2011:UBD:2132803}. Algumas das principais propostas da disciplina são introduzir noções de modelagem, transformação de caracteristicas reais em modelos de dados\cite{IF685:2017.1}, banco de dados relacional, que modela os dados em forma de relações\cite{IF685:2017.1}, banco de dados objeto-relacional, que se trata de uma extenção dos bancos de dados relacionais com suporte para dados mais complexos e referência a objetos\cite{IF685:2017.1}, e também dados Semi-estruturados, os quais não necessariamente possuem uma estrura regular ou completude\cite{IF685:2017.1}.

\section{Relevância}
Com o grande crescimento da produção e uso de dados, se torna cada vez mais importante dominar os conhecimentos dessa área, que podem ser usados em diversos setores. Do ponto de vista acadêmico também é importante dominar bem essa área pois é algo requisitado em outas disciplinas, além disso é uma forma de mostrar aplicações adiquiridas em outras disciplinas.
\begin{table}[h]
\centering
\caption{Alguns dos pontos negativos e positivos}
\label{my-label}
\begin{tabular}{l|l}
Pontos positivos                                                                                  & Pontos negativos                                                                                               \\ \hline
\begin{tabular}[c]{@{}l@{}}Possui aplicações diretas e\\ reais\end{tabular}                       & \begin{tabular}[c]{@{}l@{}}Exige muito conhecimento\\ prévio de outras disciplinas\end{tabular}                \\ \hline
\begin{tabular}[c]{@{}l@{}}Permite trabalhar com\\ grande quantidade de\\ informação\end{tabular} & \begin{tabular}[c]{@{}l@{}}Normalmente os bancos de\\ dados são uma atividade\\ meio e não um fim\end{tabular} \\ \hline
\begin{tabular}[c]{@{}l@{}}Auxilia no uso de\\ informações complexas\\ e incompletas\end{tabular} & \begin{tabular}[c]{@{}l@{}}Nem sempre leva a\\ conclusões concretas\end{tabular}                              
\end{tabular}
\end{table}
\newpage
\section{Relação com outras disciplinas}

A disciplina de gerenciamento de dados da Universidade Federal de Pernambuco tem como único pré-requisito, algoritmos e estrutura de dados. Éntretanto é pré-requisito de dez disciplinas\cite{UFPE:2013}, mas pode ser de útilidade em qualquer disciplina que necessite conhecimentos sobre modelos e bancos de dados.

\begin{table}[h]
\centering
\caption{Algumas das disciplinas relacionadas}
\label{my-label}
\begin{tabular}{|l|l|}
\hline
\begin{tabular}[c]{@{}l@{}}IF672 - ALGORITMOS E ESTRUTURA\\ DE DADOS\end{tabular}      & \begin{tabular}[c]{@{}l@{}}É pré-requisito da disciplina, além disso é de\\ extrema importância pois auxilia o programador\\ a escrever programas de maior eficiência,\\ introduzindo algoritmos que serão necessários\\ durante em gerenciamento de dados.\end{tabular}                                                                                      \\ \hline
IF669 - INTRODUÇÃO À COMPUTAÇÃO                                                        & \begin{tabular}[c]{@{}l@{}}É indiretamente pré-requisito da disciplina pois é\\ pré-requisito de algoritmos e estrutura de dados,\\ é de grande importância pois serão necessários \\ conhecimentos de programação, além disso a \\ introdução a orientação a objeto auxiliarão o \\ melhor entendimento de banco de dados \\ objeto-relacional.\end{tabular} \\ \hline
IF699 - APRENDIZAGEM DE MÁQUINA                                                        & \begin{tabular}[c]{@{}l@{}}Depende fortemente do armazenamento e\\ gerenciamento de dados.\end{tabular}                                                                                                                                                                                                                                                       \\ \hline
\begin{tabular}[c]{@{}l@{}}IF694 - BANCO DE DADOS DISTRIBUÍDOS\\ E MÓVEIS\end{tabular} & \begin{tabular}[c]{@{}l@{}}Depende dos conhecimentos obtidos em\\ gerenciamento de dados, pois é uma parte mais\\ especifica da área, introduzindo novas ferramentas\\  para o aluno que já domina o gerenciamento de\\ dados.\end{tabular}                                                                                                                   \\ \hline
IF692 - PROJETO DE BANCO DE DADOS                                                      & \begin{tabular}[c]{@{}l@{}}Tem o objetivo de utilizar os conhecimentos\\ adquiridos em gerenciamento de dados para \\ desenvolver um projeto a ser entregue para \\ um cliente real\cite{CINWIKI-IF692}.\end{tabular}                                                                                                                                                             \\ \hline
\end{tabular}
\end{table}

\bibliographystyle{alpha}
\bibliography{sample}

\end{document}