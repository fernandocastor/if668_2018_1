\documentclass[pt10]{article}

%% Language and font encodings
\usepackage[english]{babel}
\usepackage[utf8x]{inputenc}
\usepackage[T1]{fontenc}

%% Sets page size and margins
\usepackage[a4paper,top=3cm,bottom=2cm,left=3cm,right=3cm,marginparwidth=1.75cm]{geometry}

%% Useful packages
\usepackage{amsmath}
\usepackage{graphicx}
\usepackage[colorinlistoftodos]{todonotes}
\usepackage[colorlinks=true, allcolors=blue]{hyperref}

\title{IF686 - Paradigmas de Linguagens Computacionais}
\author{Lucas Nascimento Távora}

\begin{document}
\maketitle


\section{Introdução}

Um paradigma de programação fornece e determina a visão que o programador possui sobre a estruturação e execução do programa. \cite{de2006haskell} Por exemplo, em programação orientada a objetos, os programadores podem abstrair um programa como uma coleção de objetos que interagem entre si, enquanto em programação funcional os programadores abstraem o programa como uma sequência de funções executadas de modo empilhado. A disciplina de Paradigmas de Linguagens Computacionais, no CIn, visa apresentar paradigmas alternativos ao imperativo. Os alunos apresentam uma compreensão mais refinada do significado das diversas construções utilizadas em linguagens de programação modernas e uma visão crítica das características dessas linguagens. Através dessa visão crítica, deverão ser capazes de, dado um conjunto de problemas, escolher, entre as várias linguagens existentes, as mais eficazes para resolver esses problemas. É importante frisar, porém, que o curso não visa ensinar uma linguagem específica, embora tenha uma ênfase particular na linguagem Haskell, como um exemplo do paradigma funcional, e em Java como um exemplo do paradigma concorrente (com alguns exemplos de Haskell também) . O relacionamento entre paradigmas de programação e linguagens de programação pode ser complexo pelo fato de linguagens de programação poderem suportar mais de um paradigma.\cite{lipovaca2011learn} \\
    
\section{Relevância}    
A necessidade em diferenciar uma linguagem de um paradigma é o que pode mover a importância dos alunos do CIn terem no seu currículo essa cadeira. Quando se tem uma visão crítica para as demais linguagens, a fim de resolver problemas e solucionar erros a partir da eficácia e a eficiência que é subjetiva e altamente dependente da experiência, habilidade e criatividade do programador. Por um lado é possível observar como ponto positivo de se estudar essa cadeira é a gama de linguagens e de lógica que pode-se conseguir ao entrar em contato com esse ramo. Por outro lado, a superficialidade com a qual se passa pelas linguagens pode ser um gargalo a ser observado na formação do futuro cientista da computação.\cite{goetz2006java}
\pagebreak
\section{Relação com outras disciplinas}
\begin{table}[h!]
\centering
\caption{Relação com outras disciplinas}
\label{my-label}
\begin{tabular}{|l|l|}
\hline
\multicolumn{1}{|c|}{IF669 - INTRODUCAO A PROGRAMACAO} & \begin{tabular}[c]{@{}l@{}}Por ser uma cadeira inicial e apresentar \\ os primeiros conceitos de linguagem da\\ programação ela funciona com a base \\ para o entendimento de qualquer código \\ trabalhado na cadeira de "Paradigmas".\end{tabular}                                                                        \\ \hline
IF673 - LOGICA PARA COMPUTACAO                         & \begin{tabular}[c]{@{}l@{}}Nessa cadeira, pode-se notar como a\\  ideia de pensamento que se deve ter \\ ao ler ou construir um código, seja ele \\ em qual linguagem for. Isso é importante \\ para a cadeira de "Paradigmas" por fazer \\ compreender a coerência e a coesão entre\\ a linguagem e a língua.\end{tabular} \\ \hline
IF681 - INTERFACES USUARIO-MAQUINA                     & \begin{tabular}[c]{@{}l@{}}Observa-se uma intimidade entre essas \\ cadeiras a partir da dificuldade que muitos\\  alunos tem em abordar a interface \\ usuário-máquina.\end{tabular}                                                                                                                                       \\ \hline
\end{tabular}
\end{table}






\bibliographystyle{alpha}
\bibliography{sample}

\end{document}