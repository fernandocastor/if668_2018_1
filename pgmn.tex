\documentclass[a4paper]{article}

%% Language and font encodings
\usepackage[brazil]{babel}
\usepackage[utf8x]{inputenc}
\usepackage[T1]{fontenc}
\fontsize{10}{10}\selectfont

%% Sets page size and margins
\usepackage[a4paper,top=3cm,bottom=2cm,left=3cm,right=3cm,marginparwidth=1.75cm]{geometry}

%% Useful packages
\usepackage{amsmath}
\usepackage{graphicx}
\usepackage[colorinlistoftodos]{todonotes}
\usepackage[colorlinks=true, allcolors=blue]{hyperref}

\title{IF781 - Empreendimentos em Informática}
\author{Pablo Neves}

\begin{document}
\maketitle
\section{Introdução}
\subsection{Disponibilidade}
Disponível para os alunos de engenharia e ciência da computação
do Centro de Informática da UFPE, a disciplina Empreendimentos
em Informática é ofertada como eletiva a partir do 6º período 
dos cursos e não possui pré-requisitos.
\subsection{Perfil e Conteúdo}
\label{conteudo}
Empreendimentos em informática (IF781) faz parte do \href{http://www.cin.ufpe.br/~graduacao/reforma/a_perfil_empreendedor_informatica.htm}{perfil empreendedor em informática}. Os conteúdos ministrados nessas disciplinas são voltados para o mercado e suas oportunidades, como por exemplo: pesquisas de mercado; organização de equipes; monitoramento de fluxo de caixa; planejamento; lucratividade; dentre outros. Além dos livros utilizados no curso (\cite{manualempreendedorismo2003} e \cite{planejamentosestrategico}) o site da disciplina conta com artigos(\cite{culturainovacao}; \cite{ideacao}; \cite{processoscriativos}; \cite{7answers}) para auxiliar no entendimento e fixação do conteúdo.
\section{Relevância}
Assim como outras disciplinas do perfil empreendedor em informática, Empreendimento em informática (IF781) é de suma importância para o currículo de um empreendedor em TI, profissional de alta demanda no mercado em praticamente qualquer lugar do mundo, principalmente em Recife que se destaca pelas oportunidades oferecidas desde a criação e ampliação do \href{https://www.cesar.org.br/}{CESAR.} Os conceitos e experiências passados na disciplina (\ref{conteudo}.) formam o profissional capacitado para planejar, gerenciar e inovar, ambientado com a linguagem e aplicações no mercado de trabalho.
\section{Relação com outras disciplinas}
Como eletiva de perfil a disciplina Empreendimentos em informática possui outras disciplinas relacionadas com seu conteúdo e pertencentes ao mesmo perfil curricular, como mostrado na tabela \ref{tabela}. \pagebreak

\begin{table}[h]
\centering
\caption{Outras disciplinas do perfil Empreendedor em Informática}
\label{tabela}
\begin{tabular}{|l|l|}
\hline
Disciplinas Relacionadas                                                                &                                                                                                                                                                                                                                                                                                                                                                                                                                                                                               \\ \hline
Gestão de Negócios (IF783)                                                              & \begin{tabular}[c]{@{}l@{}}Essa disciplina tem como objetivo que o aluno aprenda conceitos e técnicas\\ de gestão de empreendimento em TI. Os conteúdo aprendido em Empreen-\\ dimentos em Informática (IF781) se relacionam  diretamente  com  o conte\\ údo dessa  disciplina,  que, como sugere o nome, passa  pela Gestão de um \\ Negócio.  Conteúdos  como: gestão de pessoas; escolha de produtos para o\\ mercado; finanças, complementam o perfil empreendedor do aluno.\end{tabular} \\ \hline
\begin{tabular}[c]{@{}l@{}}Economia para Empreendedores\\ (IF780)\end{tabular}          & \begin{tabular}[c]{@{}l@{}}Curso  que  visa suprir  a  base  teórica para análise e situações de mercado \\ como: tipos de concorrência; monopólio;  jogo de barganha; dentre  outros.\\  Análise de mercado e situações  é fundalmental para  preparar  o  aluno de\\  para ser bem sucessido.\end{tabular}                                                                                                                                                                                   \\ \hline
\begin{tabular}[c]{@{}l@{}}Contabilidade Financeira \\ e Gerencial (IF784)\end{tabular} & \begin{tabular}[c]{@{}l@{}}Mais uma  disciplina do perfil  empreendedor em informática, nesse  curso\\ o aluno deve apresentar um projeto com  a análise de  sua viabilidade   eco-\\ nômico -financeira. Utiliza-se muito do  aprendido  em outras disciplinas do\\ perfil empreendedor como: fluxo de caixa; pesquisa de mercado; gestão.\end{tabular}                                                                                                                                       \\ \hline
\end{tabular}
\end{table}



\bibliographystyle{unsrt}
\bibliography{pgmn}

\end{document}