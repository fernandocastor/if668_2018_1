\documentclass[a4paper]{article}

%% Language and font encodings
\usepackage[brazil]{babel}
\usepackage[utf8x]{inputenc}
\usepackage[T1]{fontenc}

%% Sets page size and margins
\usepackage[a4paper,top=3cm,bottom=2cm,left=3cm,right=3cm,marginparwidth=1.75cm]{geometry}

%% Useful packages
\usepackage{float}
\usepackage{amsmath}
\usepackage{graphicx}
\usepackage[colorinlistoftodos]{todonotes}
\usepackage[colorlinks=true, allcolors=blue]{hyperref}

\title{IF675 - Sistemas Digitais}
\author{Maria Clara Dionisio Amaral Gois}

\begin{document}
\maketitle

\begin{abstract}
A cadeira de Sistemas Digitais, lecionada pelo professor Manoel Eusébio de Lima na Universidade Federal de Pernambuco, apresenta carga horária de 75 horas e visa ensinar aos alunos conceitos sobre circuitos lógicos digitais, abrangendo desde sistemas poucos complexos até os mais complexos. Ademais, busca ensinar aos alunos sobre a eficiência da informação digital na manipulação de técnicas para o processamento da informação.
\end{abstract}

\section*{Introdução}
A cadeira de Sistemas Digitais dada no CIn(Centro de Informática), na Universidade Federal de Pernambuco, intenta ensinar aos alunos dos cursos de Ciências da Computação e de Engenharia de Computação no segundo período de curso como entender o funcionamento de computadores digitais, como desenvolver projetos de “circuitos integrados” voltados para Embedded systems(Sistemas embarcados) e como utilizar técnicas modernas que permitam desenvolver sistemas de para o processamento de informações. Com a finalidade de elevar o aprendizado dos universitários, são propostos projetos e o uso de ferramentas de CAD(Computer Aided Design) e, ao longo do curso, os alunos podem utilizar esquemas ou HDLs(VHDL-Very High Speed Integrated Circuit Hardware description Language) para usarem a ferramenta do CAD.

\begin{figure}[H]
\centering
\includegraphics[width=0.55\textwidth]{circuito_digital.png}
\caption{\label{fig:Circuito} Exemplo de um sistema digital.}
\end{figure}

\section*{Relevância}
A cadeira de Sistemas Digitais da UFPE(IF675) está no currículo do curso de Ciências da Computação, pois essa cadeira aborda sobre o que é um sistema digital, sua funcionalidade, como criar sistemas digitais e para que esses sistemas são utilizados. Essa cadeira é fundamental para os graduandos de Ciências da Computação e de Engenharia de Computação, pois os sistemas digitais são importantes para as áreas de tecnologia, já que são dispositivos programados para realizar uma funcionalidade a partir de determinado comando. Cada sistema digital utiliza valores discretos, descontínuos(0 e 1) e possui duas entradas, mas apenas uma saída e, a partir dos valores recebidos(0 ou 1), produz uma relação entre a entrada e a saída e, assim, consegue fazer com que uma função seja realizada. Ademais, os sistemas digitais apresentam vantagens como facilidade de serem projetados, devido ao uso de tensões "fortes" e "fracas", em vez de necessitar dos valores reais; facilidade armazenamento de informação; maior precisão e exatidão; menos ruídos, quando comparados aos sistemas analógicos; e uma elevada adequação à integração e, por isso, é importante importante ensinar Sistemas Digitais.\\

\section*{Relação com outras disciplinas}
\begin{table}[H]
\caption{Cadeiras Relacionadas à cadeira de Sistemas Digitais}
\centering
\label{my-label}
\begin{tabular}{|l|l|}
\hline
IF674- Infra-estrutura de Hardware        & \begin{tabular}[c]{@{}l@{}}Essa cadeira se relaciona com a cadeira de Sistemas  Digitais,\\ pois aborda sobre estruturas de um sistemas, circuitos e  estu-\\ do de  casos. Dessa forma, conceitos  utilizados em  Sistemas \\ Digitais ajudariam no entendimento dessa.\end{tabular}             \\ \hline
IF732- Projetos de Sistemas Embutidos     & \begin{tabular}[c]{@{}l@{}}Essa cadeira se relaciona com a cadeira de Sistemas  Digitais,\\ pois aborda sobre sistemas hardware e a construção de dispo-\\ sitivos, logo, utiliza parte dos conhecimentos de Sistemas Di-\\ gitais.\end{tabular}                                                  \\ \hline
IF729- Protótipos de Circuitos Integrados & \begin{tabular}[c]{@{}l@{}}Essa cadeira se relaciona com a cadeira de Sistemas  Digitais,\\ pois aborda sobre tecnologia para implementação de circuitos\\ integrados  e de alta integração. Assim,  os conceitos de Siste-\\ mas Digitais facilitaria o aprendizado dessa cadeira.\end{tabular} \\ \hline
\end{tabular}
\linebreak
\linebreak
\linebreak
\end{table}



\bibliographystyle{alpha}
\bibliography{mcdag}

\cite{ercegovac2000introduccao}
\cite{booth1984introduction}
\cite{taub1984circuitos}\\

\url{https://commons.wikimedia.org/wiki/File:NAND_from_NOR.svg?\\uselang=pt}


\end{document}