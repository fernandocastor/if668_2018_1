\documentclass[10pt]{article}

\usepackage[english]{babel}
\usepackage[utf8x]{inputenc}

\usepackage{lipsum}
\usepackage{fullpage}

\usepackage[colorlinks=true, allcolors=blue]{hyperref}

\title{IF975 - Redes de Computadores}
\author{Maria Beatriz Campelo Bezerra}
\date{11 de Maio de 2018}

\begin{document}
\maketitle
\section{Introdução}

Uma rede de computadores consiste em um conjunto de máquinas interconectadas, o que permite a troca de dados e informações entre si. Dessa forma, os profissionais da área são responsáveis por configurar, instalar, testar e projetar redes de computadores dentro de empresas. Os pricipais objetivos da disciplina (IF975 - Redes de Computadores) são introduzir conceitos, protocolos e arquiteturas para redes de computadores e, dessa forma, capacitar o aluno para a realização de procedimentos de maneira adequada, além de apresentar as pricipais tendências do mercado na área. 

\section{Relevância}

Após a criação dos microcomputadores, as pessoas passaram a ter a possibilidade de realizar diversas tarefas e atividades utilizando os recursos computacionais. No entanto, por não serem interconectados, os microcomputadores tinham a limitação do distanciamento entre usuários. À partir da criação da rede de computadores, a barreira da distância foi rompida e as tecnologias de rede extandiram-se vertiginosamente com os novos produtos nos anos 80. A capacidade de antingir inúmeras pessoas é um ponto de extrema importância quando se visa economizar tempo e dinheiro, principalmente num mundo globalizado, com empresas multinacionais e intercâmbio de informações entre praticamente todos os países do globo. As redes de computadores também contribuiram para o crescimento do número de empregos, pois é uma área de atuação bastante abrangente e versátil, que vai desde residências familiares até grandes multinacionais, por ser capaz de se moldar às necessidades do usuário. Com o advento da internet das coisas, teoricamente qualquer coisa (como geladeira, lâmpadas, televisão, entre outros) pode estar conectada à internet e se comunicar em rede. Sendo assim, o que torna "Redes de Computadores" uma disciplina tão fundamental é o fato de que ela introduz os alunos em uma área que tem o poder de fazer com que diversos usuários compartilhem informações e ideias independentemente da distância.



\section{Relação com outras disciplinas}
\begin{table}[!ht]
\centering
\begin{tabular}{|c|c|}
\hline
\textbf{Disciplina} & \textbf{Relação} \\\hline
IF678 - Infra-estrutura de comunicação & Na disciplina de infra-estrutura de comunicação se estuda \\& à fundo redes de computadores, Web, segurança \\& de redes, entre outros assuntos convergentes.
\\\hline
IF679 - Informática e Sociedade & Na disciplina de informática e sociedade se estuda vários aspectos\\& da relação ente homem, máquina e economia, o que pode ser\\& relacionado com a a globalização do conhecimento por meio \\&da rede.\\
\hline
\end{tabular}
\end{table}

\begin{table}[!ht]
\centering
\begin{tabular}{|c|l|}
\hline
IF690 - História e futuro da Computação & Na cadeira de História e futuro da Computação se \\& estudam alguns eventos que impulsionaram a história da\\& tecnologia, dentre eles a criação da rede de computadores
\\\hline
LE530 - Inglês para Computação &O inglês é importante para todas as áreas da computação\\& pois a maior parte dos programas usam termos em inglês,\\& vários livros e artigos científicos da área estão neste\\& idioma e as empresas multinacionais renomadas utilizam\\& o inglês interna e externamente. 
\\\hline
IF747 - Top. Avanc. Redes de Computadores &  Como já explicitado no nome, a cadeira de Top. Avanc.\\& Redes de Computadores aprofunda os estudos iniciados\\& em Redes de Computadores. \\
\hline
\end{tabular}
\end{table}
\section{Referências das Disciplinas}
As referências usadas nos sites da cadeira de Redes de computadores são os livros \cite{redes1} e \cite{redes2}.
\bibliographystyle{alpha}
\bibliography{redes}

\end{document}