\documentclass[a4paper,10pt]{article}

%% Language and font encodings
\usepackage[brazil]{babel}
\usepackage[utf8x]{inputenc}
\usepackage[T1]{fontenc}


%% Sets page size and margins
\usepackage[a4paper,top=3cm,bottom=2cm,left=3cm,right=3cm,marginparwidth=1.75cm]{geometry}

%% Useful packages
\usepackage{amsmath}
\usepackage{graphicx}
\usepackage[colorinlistoftodos]{todonotes}
\usepackage[colorlinks=true, allcolors=blue]{hyperref}

\title{IF702 - Redes Neurais}
\author{Gabriel de Oliveira Sousa}

\begin{document}
\maketitle



\begin{figure}[h]
\centering
\includegraphics[width=0.6\textwidth]{nn.png}
\caption{\label{fig:redes} \href{http://www.latex-tutorial.com}{CC BY 1.0}. }
\end{figure}

\section{Introdução}

Redes Neurais são sistemas computacionais inspirados no cérebro humano, em outras palavras, tentar trazer nossas capacidades cognitivas(criatividade,apredizagem, etc ...) para o computador. A disciplina foca-se em:

\begin{enumerate}
\item MODELOS DE NEURÔNIOS: o neurônio e sua representação matemática.
\item ARQUITETURA E APRENDIZADO NAS REDES NEURAIS: Os diferentes algoritmos e representações matemática tais como:perceptrons e multilayer perceptrons,modelos de Hopfield e ART.
\item APLICAÇÕES: consiste em fazer um projeto aplicando os conceitos vistos. 
\end{enumerate}

\section{Relevância}



Ao longo dos últimos anos houve uma explosão no uso de algoritmos de redes neurais e aprendizado de máquina, puxado por grandes empresas como Google,IBM e Microsoft. Isso se deve ao poder de aplicação em várias áreas como na saúde, na economia e na sociedade. E com a explosão da geração de dados que está cada vez mais imensurável, é necessário que tenhamos conhecimentos para processar esses dados e trazer conhecimento , que sem uso de tais técnicas se torna quase impossível. É nesse momento que entra a disciplina IF702 - Redes Neurais, preparar os alunos do Cin é vital para que eles possam encarar os desafios de agora e do futuro.

\subsection{Pontos positivos}
\begin{enumerate}
\item oferece boa base matemática
\item apresenta aplicações em várias áreas
\item conteúdo completo
\end{enumerate}
\subsection{Pontos negativos}
\begin{enumerate}
\item poderia ter uma carga horária prática maior

\end{enumerate}

\section{Relação com outras disciplinas}


\begin{center}
    \begin{tabular}{ | l | p{5cm} |}
    \hline
    código e nome da disciplina & relação\\ \hline
    IF684- SISTEMAS INTELIGENTES  & redes neurais é uma sub-área da inteligência artifical, RNAs também utilizam do vários conceitos vistos em IF684  \\ \hline
    
    IF752- ANALISE IMAG. VISAO COMPUTACIONAL  & redes neurais grandes aplicações em processamento de imagens e visão computacional, já que carros autonômos , por exemplos, faz o uso dos conceitos de deep learning \\ \hline
    
    IF699- APRENDIZAGEM DE MAQUIN  & duas áres bem relacionadas, já que duas cadeiras compartilham vários conceitos chaves\\ \hline
    
    IF754- COMPUTACAO MUSICAL   & redes neurais podem ser usadas para gerar músicas \\ \hline
    IF796- MINERACAO DA WEB & termos como processamento de linguagem natural é capaz de processar todo conteúdo da web, técnicas modernas faz o uso de deep learning \\ \hline
    IF797- OTIMIZACA & redes neurais para ser eficientes precisam ser otimizadas, conceitos vistos em IF797 são fundamentais para o desenvolvimento efetivo das mesmas\\ \hline
    
    \end{tabular}
\end{center}


\section{Referências}
\begin{enumerate}
\item Neural Computing : An Introduction. R. Beale, T. Jackson. (1990).
\item Redes Neurais Artificiais: Teoria e Aplicações. Braga, A.P, Ludermir, T.B, Carvalho, A. F. (2000)
\item  Neural Networks and Deep Learning. Michael Nielsen. (2017).
\end{enumerate}





\bibliography{sample}

\end{document}