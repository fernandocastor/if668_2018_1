\documentclass[a4paper,10pt]{article}

%% Language and font encodings
\usepackage[brazil]{babel}
\usepackage[utf8x]{inputenc}
\usepackage[T1]{fontenc}

%% Sets page size and margins
\usepackage[a4paper,top=3cm,bottom=2cm,left=3cm,right=3cm,marginparwidth=1.75cm]{geometry}

%% Useful packages
\usepackage{amsmath}
\usepackage{graphicx}
\usepackage[colorinlistoftodos]{todonotes}
\usepackage[colorlinks=true, allcolors=blue]{hyperref}

\title{IF682- Engenharia Software e Sistemas}
\author{Johnny Herbert Muniz Nunes}

\begin{document}
\maketitle

\section{Introdução}

Engenharia Software de Sistemas é uma disciplina que é voltada para os aspectos
da produção de software. As principais atividades da engenharia de software é a
especificação de software, desenvolvimento de software, validação de software e
evolução de software. A engenharia de software está preocupada em desenvolver o
software o mais próximo possível do que foi solicitado de forma mais rápida e
segura. A disciplina promove atividades que requerem trabalho em equipe executável, pois se refere a software
(sistemas) desenvolvidos por grupos ao invés de indivíduos.\cite{PaginaDisciplina}


\begin{figure}[h]
\centering
\includegraphics[width = 0.3\textwidth]{EngenhariaSoftware.jpg}
\caption{\label{fig:Engenharia_Software}Engenharia Software e Sistemas.}
\end{figure}
%Link: https://commons.wikimedia.org/wiki/File:EngenhariaSoftwareFaesi.jpg
%Licença:
%CC0 1.0 Universal (CC0 1.0)
%https://creativecommons.org/publicdomain/zero/1.0/deed.en


\section{Relevância da disciplina}

\subsection{Deveria está no currículo de ciência da computação?}

Essa disciplina é muito importante na matriz curricular do curso de ciência da computação, pois nela iremos abordar vários métodos de desenvolvimento de software e aprender como planejar e gerenciar de forma eficiente a elaboração de um projeto.

\begin{figure}[b]
\centering
\includegraphics[width=0.3\textwidth]{Duvida.png}
\caption{\label{fig:Duvida}Dúvida.}
\end{figure}
%Link: https://commons.wikimedia.org/wiki/Category:Doubt#/media/File:Mr_Pipo_Think_03.svg
%Licença: 
%(CC BY-SA 3.0)
%https://creativecommons.org/licenses/by-sa/3.0/deed.en
\subsection{Disciplinas relacionadas}

\begin{table}[!h]
\centering
\label{my-label}
\begin{tabular}{|l|l|}
\hline
IF661- Tabralho de Graduação                           & \begin{tabular}[c]{@{}l@{}}
No trabalho de graduação é onde se\\
aplica todo conhecimento adquirido\\
no curso, e nele temos que aplicar\\
o conhecimento de engenharia de\\
software.
\end{tabular}                        \\ \hline
IF722- Tópicos Avançados Eng. Software                 & \begin{tabular}[c]{@{}l@{}}
Em tópicos avançados de eng. software\\
aprende-se de forma mais específica e\\
aprofundada os conceitos de engenharia\\
de software.\end{tabular}          \\ \hline
IF723- Sem. Engenharia de Software Ling. Computacional & \begin{tabular}[c]{@{}l@{}}
Em sem. engenharia de software ling. \\
computacionais apresentam-se vários\\ 
seminários voltados a engenharia de\\
software, ajudando assim o aprofunda-
\\mento da área.
\end{tabular}        \\ \hline
IF672- Algorítimos e estruturas de dados               & \begin{tabular}[c]{@{}l@{}}
Algorítimos e estrutura de dados é\\
usado na engenharia de software, pois\\
nesta disciplina começam as preocupa-\\
ções em fazer códigos otimizados.\end{tabular} \\ \hline
\end{tabular}
\caption{Disciplinas relacionadas.}\cite{UFPE}
\end{table}
\bibliographystyle{plain}
\bibliography{ref}

\end{document}