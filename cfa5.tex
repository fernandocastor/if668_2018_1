\documentclass[a4paper, 10pt]{article}

%% Language and font encodings
\usepackage[brazil]{babel}
\usepackage[utf8x]{inputenc}
\usepackage[T1]{fontenc}

%% Sets page size and margins
\usepackage[a4paper,top=3cm,bottom=2cm,left=3cm,right=3cm,marginparwidth=1.75cm]{geometry}

%% Useful packages
\usepackage{amsmath}
\usepackage{graphicx}
\usepackage[colorinlistoftodos]{todonotes}
\usepackage[colorlinks=true, allcolors=blue]{hyperref}

\title{IF674 -INFRA-ESTRUTURA DE HARDWARE}
\author{Caio Fazio Ayres}


\begin{document}
\maketitle
\begin{figure} [h]
\centering
\includegraphics[width=\textwidth]{ca.png}
\caption{Public domain, from Wikimedia Commons}
\end{figure}
\section{Introdução}

A disciplina de Infraestrutura de Hardware faz parte da área de redes e tem seu foco em dar ao aluno uma visão geral dos componentes de um computador, como o processador,o sistema de memória, dispositivos de entrada e saída, entre outros, demonstrando e fundamentando o funcionamento desses componentes seja pelo projeto de uma versão simplificada do componente ou pela utilização de ferramentas de simulação. Para isso os alunos contam com uma carga horaria de 75 horas entre aulas teóricas e aulas práticas. 

\section{Relevância}

Saber como um computador funciona é extremamente essencial para qualquer profissional na área da computação, pois permite ao profissional saber suas capacidades e limitações e com isso, o profissional sabe como programar da maneira mais eficiente possível, explorando suas capacidades e respeitando seus limites.

\subsection{Pontos positivos}
\begin{itemize}
\item Permite ao estudante de ciência da computação, ter contato também com o hardware.
\item É importante para qualquer profissional ligado a computação.
\end{itemize}
\subsection{Pontos negativos}
\begin{itemize}
\item Para alunos sem interesse em hardware, pode se tornar uma disciplina problemática.
\end{itemize}
\section{Relação com outras disciplinas}
\begin{table}[h]
\centering
\caption{Relações interdisciplinares}
\label{my-label}
\begin{tabular}{|l|p{8cm}|} \hline

\textbf{Disciplina}                                     & \textbf{Relação}                                                                                                                                                                                                                                                                                                                                                                                                  \\\hline
Programação em geral                                    & A disciplina de Infraestrutura de Hardware mostra ao aluno como funciona um computador, suas limitações e capacidades, e programar sabendo disso é fundamental para maximo desempenho do software.                                                                                                                                                                                                                \\ \hline
Infraestrutura de Software / Infraestrutura de Hardware & Juntas as disciplinas de Infraestrutura de Software, Infraestrutura de Hardware e Infraestrutura da Comunicação formam a base para a grande maioria de sistemas de computadores, pois uma é o complemento da outra, enquanto uma trata do hardware, a outra trata da comunicação software-hardware (Infraestrutura de Software) e outra trata sobre como funciona a comunicação entre computadores pela internet. \\\hline
\end{tabular}
\end{table}

\nocite{dj}
\nocite{ws}
\bibliographystyle{alpha}
\bibliography{cfa5}

\end{document}