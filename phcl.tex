\documentclass[a4paper]{article}

%% Language and font encodings
\usepackage[portuguese]{babel}
\usepackage[utf8x]{inputenc}
\usepackage[T1]{fontenc}
\fontsize{10}{10}\selectfont
%% Sets page size and margins
\usepackage[a4paper,top=3cm,bottom=2cm,left=3cm,right=3cm,marginparwidth=1.75cm]{geometry}

%% Useful packages
\usepackage{amsmath}
\usepackage{graphicx}
\usepackage[colorinlistoftodos]{todonotes}
\usepackage[colorlinks=true, allcolors=blue]{hyperref}

\title{IF702 - Redes Neurais}
\author{Pedro Caminha}
\begin{document}
\maketitle



\section{Introdução}

\subsection{Características de Cadeira}

A cadeira Redes Neurais é de caráter eletivo, podendo ser paga por graduandos do curso de Ciência da computação a partir do 6º período. A mesma não possui pré-requisitos e co-requisitos e é ministrada pelo eminente Professor Germano Vasconcelos, quando é ofertada, geralmente no segundo semestre do ano.

\subsection{Visão Geral}

A disciplina se encaixa nas áreas de Inteligência Artificial e Biocomputação, trazendo uma visão sobre os modelos de Hopfield, Kohonen e ART, além de redes recorrentes e auto-organizáveis. A cadeira conta com um projeto de conclusão que tem por objetivo principal a análise de uma Rede Neural no MATLAB(plataforma de ensino voltado ao cálculo numérico).
\section{Importância das Redes Neurais}

Primeiramente, as Redes Neurais são definidas como uma técnica para captação de valores de entrada inspiradas no sistema nervoso central, de acordo com o livro ``Redes neurais artificiais: teoria e aplicações\cite{braga2000redes}''.

Atualmente, as Redes Neurais apresentam um papel de destaque dentro da computação, como foi comprovado pelo Hyper Cycle\cite{hypercycle17} publicado pela Gartner em 2017, ``AI technologies will be the most disruptive class of technologies over the next 10 years due to radical computational power, near-endless amounts of data and unprecedented advances in deep neural networks'' (Tecnologias de IA serão a classe tecnológica mais disruptiva nos próximos 10 anos, devido ao imenso poder computacional, quantidades quase infinitas de dados e avanços sem precedentes em redes neurais profunda).

Além disso, as Redes Neurais possuem grande aplicabilidade, como por exemplo robôs que desarmam bombas e softwares de reconhecimento de voz. 

 A figura \cite{Artificial_neural_network} seguir, é um exemplo de uma Rede Neural, onde as setas representam as entradas, os círculos os neurônios e links ocultos e no fim, a saída.
 
 \begin{figure}
\centering
\includegraphics[width=0.40\textwidth]{Artificial_neural_network.png}
\label{fig:Exemplo de Rede Neural}
\end{figure}

\section{Relacionamento com outras cadeiras}
\begin{table}[h]
\centering

\begin{tabular}{|l|p{9.0cm}|}\hline
Cadeira & Como se relaciona \\\hline
IF672- Algorítimos e Estruturas de Dados & As Redes Neurais tratam-se de algorítmos complexos e tem por objetivo-meio solucionar os problemas que requerem alto grau de processamento, objetivo similar ao da disciplina IF672.  \\\hline
IF699- Aprendizagem de Máquina & Além de ser a área onde as Redes Neurais se inserem, a cadeira tem por objetivo, o estudo de algorítimos para análise de dados e reconhecimentos de padrões .\\\hline
IF684- Sistemas Inteligentes & A cadeira baseia-se no aprendizado de máquina e nas escolhas mais eficientes, elementos fundamentais às Redes Neurais. \\\hline


\end{tabular}
\textit{\caption{\label{tab:widgets}Como a cadeira de Rede Neurais se relaciona com outras cadeiras.}
}\end{table}
\section{Referências Bibliográficas da disciplina}
Estas são as referências listadas no site da Disciplina:
\begin{enumerate}
\item Neural Computing : An Introduction. R. Beale, T. Jackson. (1990).\cite{beale1990neural}
\item Redes Neurais Artificiais: Teoria e Aplicações. Braga, A.P, Ludermir, T.B, Carvalho, A. F. (2000)\cite{braga2000redes}
\item Neural Networks and Deep Learning. Michael Nielsen. (2015).\cite{nielsen2015neural}  
\item Neural Computation: A Comprehensive Foundation. Simon Haykin. (2004).\cite{haykin2004comprehensive}

\end{enumerate}


\bibliographystyle{unsrt}
\bibliography{phcl}


\end{document}