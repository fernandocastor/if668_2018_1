\documentclass[a4paper]{article}

%% Language and font encodings
\usepackage[brazil]{babel}
\usepackage[utf8x]{inputenc}
\usepackage[T1]{fontenc}

%% Sets page size and margins
\usepackage[a4paper,top=3cm,bottom=2cm,left=3cm,right=3cm,marginparwidth=1.75cm]{geometry}

%% Useful packages
\usepackage{float}
\usepackage{amsmath}
\usepackage{graphicx}
\usepackage[colorinlistoftodos]{todonotes}
\usepackage[colorlinks=true, allcolors=blue]{hyperref}

\title{IF672 - Algoritmos e Estruturas de Dados}
\date{\vspace{-5ex}}
\author{Maria Giovana Accioly Santos}

\begin{document}
\maketitle

\begin{abstract}
A cadeira IF672 - Algoritmos e Estruturas de Dados é lecionada no CIn da Universidade Federal de Pernambuco pelos professores Paulo Fonseca e Katia Guimarães. É uma cadeira obrigatória com 75 horas de carga horária teórica, concedendo 5 créditos. Sua finalidade principal é fazer com que o aluno aprenda a construir algoritmos e estruturas de dados bastante eficientes.
\end{abstract}

\section*{Introdução}

A ementa da cadeira se baseia nos seguintes tópicos: Introdução à análise de algoritmos, elementos de estruturas de dados, algoritmos de busca, algoritmos geométricos, noções de buscas por exaustão e problema np completos.
Análise de algoritmos é o estudo da complexidade dos algoritmos, que serve para determinar o que é necessário para executar um certo algoritmo, como por exemplo o tempo útil para sua execução ou a quantidade de memória a ser usada. Os elementos das estruturas de dados podem existir de diversas maneiras, sendo em sequencias, de tipos diferentes, possuindo ponteiros, etc. Algoritmos de busca sāo estratégias de armazenamento de dados, que servem para encontrar um certo dado numa estrutura. Algoritmos geométricos são partes fundamentais da Computação Geométrica e para resoluções de seus problemas utilizando a menor quantidade possível de operações simples sobre os elementos geométricos. Noções de buscas por exaustāo e problema np completos são partes teóricas da computação, que envolvem problemas aparentemente impossíveis de serem resolvidos em um curto tempo.
	Algoritmos e Estruturas de Dados passa por diversas áreas da computaçāo e com diversas possibilidades de aplicação, pois algoritmos servem para utilizar dados e organizá-los em uma estrutura que vai apresentar uma determinada forma com suas vantagens estratégicas.

\section*{Relevância}

A Estrutura de Dados (ED) serve para organizar os dados em uma máquina para que possam ser usados da maneira mais eficiente possível, melhorando suas modificações e buscas. É muito importante para minimizar a memória RAM utilizada e para deixar o código mais simples e curto. Portanto, tem extrema relevância para a praticidade da produção de códigos na programação, que é um dos pilares da Ciência da Computação e de muitas cadeiras do curso.

\section*{Relação com outras disciplinas}

\begin{table}[H]
\centering
\caption{Relações entre disciplinas}
\label{my-label}
\begin{tabular}{|l|l|}
\hline
IF670 - Matemática Discreta                 & \begin{tabular}[c]{@{}l@{}}Essa disciplina é fundamental para conhecer o básico\\ teórico dos elementos usados para construir os algoritmos,\\ além de agregar conhecimentos lógicos sobre teoremas e\\ equações usadas para fazer as estruturas de dados\end{tabular}                 \\ \hline
IF685 - Gerenciamento de dados e Informação & \begin{tabular}[c]{@{}l@{}}Algoritmos e Estruturas de Dados é um pré-requisito para\\ essa disciplina, pois é fundamental entender e fazer uma ED\\ para poder gerenciar seus dados e informações\end{tabular}                                                                         \\ \hline
IF689 - Informática Teórica                 & \begin{tabular}[c]{@{}l@{}}Algoritmos e Estruturas de Dados é um pré-requisito para    \\ cursar essa cadeira, pois sua base é fundamental para o en-\\ tendimento da Informática Teórica, já que o assunto dado é\\ extendido.\end{tabular}                                           \\ \hline
IF682 - Engenharia de Software e Sistemas   & \begin{tabular}[c]{@{}l@{}}Algoritmos e Estrutura de Dados é um pré-requisito para a\\ essa disciplina, já que para criar softwares e sistemas é ne-\\ cessário também criar algoritmos e ED eficientes, que sejam \\ utilizados para a criação dos softwares e sistemas.\end{tabular} \\ \hline
IF684 - Sistemas Inteligentes               & \begin{tabular}[c]{@{}l@{}}Algoritmos e Estrutura de Dados é um pré-requisito para a \\ disciplina, já que para a criação desses sistemas são utilizados\\ algoritmos e ED que possibilitem uma função específica e efi-\\ ciente.\end{tabular}                                        \\ \hline
IF669 - Introduçāo à Programaçāo            & \begin{tabular}[c]{@{}l@{}}Aprender a programar e a lógica da programação são muito\\ importantes para começar a criar algoritmos e estruturas de\\ dados\end{tabular}                                                                                                                 \\ \hline
\end{tabular}
\end{table}


\cite{levitin2012introduction}
\cite{dasgupta2006umesh}
\cite{shaffer2013data}
\cite{cormen2009introduction}


\bibliographystyle{alpha}
\bibliography{mgas} 


\end{document}