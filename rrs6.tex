\documentclass[10pt,a4paper]{article}

%% Language and font encodings
\usepackage[brazil]{babel}
\usepackage[utf8x]{inputenc}
\usepackage[T1]{fontenc}

%% Sets page size and margins
\usepackage[a4paper,top=3cm,bottom=2cm,left=3cm,right=3cm,marginparwidth=1.75cm]{geometry}

%% Useful packages
\usepackage{amsmath}
\usepackage{graphicx}
\usepackage[colorinlistoftodos]{todonotes}
\usepackage[colorlinks=true, allcolors=blue]{hyperref}

\title{IF738 - Redes de Computadores}
\author{Rafael Rodrigues da Silva}

\begin{document}
\maketitle

\section{Introdução}
\textbf{IF738 - Redes de Computadores} É cadeira do tipo eletiva, é vista normalmente pelos alunos de Ciência da Computação da Universidade Federal de Pernambuco. Porém há pré-requisito, na qual o graduando que tenha interesse em cursar a disciplina, tenha cursado a cadeira IF678-Infra-Estrutura de Comunicação. A cadeira de Redes atualmente é ministrada pelo professor Djamel Sadok e tem como carga horária 60h. A disciplina visa fornecer um alicerce na área de Redes de Computadores, em sua parte avançada. Por isso, o aluno estuda inicialmente na cadeira:

\begin{itemize}
\item \textbf{Introdução as Redes de Comunicação.}
\item \textbf{Modelo de Referência OSI.}
\item \textbf{Camada Física (Técnicas de Transmissção Analógica e Digital; Técnicas de Multiplexação FDM e TDM; Rede Digital de Serviços Integrados).}
\item \textbf{Sub-camada de Acesso ao Meio.}
\item \textbf{Redes Locais e Metropolitanas.}
\item \textbf{Camada de Enlace de Dados.}
\item \textbf{Dimensionamento de Redes.}
\end{itemize}

Vale salientar, que há na cadeira exercicios, provas e um seminário com o tema escolhido pelo aluno e aprovado pelo professor. Em que o grupo do seminário é formado por 4 graduandos, no qual deve ser preparada uma apresentação a ser feita em sala de aula, assim como um relatório técnico.

\section{Relevância}
O graduando que cursa a cadeira tem seus horizontes acadêmico ampliado. Tendo em vista, que está cadeira gera um diferencial no curriculo acadêmico do aluno. Pois, o conhecimento que a cadeira proporciona na área de redes, ajudará a aluno a ser um profissional cada vez mais completo. Contudo, a cadeira tem seus pontos negativos e positivos, os mesmo estão listados abaixo.

\subsection{Pontos positivos}
\begin{itemize}
\item Enriquece o curriculo do graduando, ajudando o aluno a ser destacar no âmbito profissional.
\item Vários exercícios de fixação, que ajudam o aluno a assimilar o assunto rapidamente.
\item Projetos em equipes, que contribuem para a absorção do assunto e o desenvolvimento do trabalho em equipe. 
\end{itemize}
\subsection{Pontos negativos}
\begin{itemize}
\item Pouca carga horária de aulas.
\end{itemize}

\section{Relação com outras disciplinas}
Por ser eletiva a cadeira tem relações com outras cadeiras, abaixo estão listados as abaixo.

\begin{tabular}{lllll}
\cline{1-2}
\multicolumn{1}{|l|}{Disciplina}                               & \multicolumn{1}{l|}{Relação}                                                       &  &  &  \\ \cline{1-2}
\multicolumn{1}{|l|}{IF678 - Infra-Estrutura de Comunicação}   & \multicolumn{1}{l|}{Pré-requisito para cursar IF738 - Redes de Computadores.}      &  &  &  \\ \cline{1-2}
\multicolumn{1}{|l|}{IF747 - Top.Avanc. Redes de Computadores} & \multicolumn{1}{l|}{IF738 - Redes de Computadores é pré-requisito para a cadeira.} &  &  &  \\ \cline{1-2}
                                                               &                                                                                    &  &  & 
\end{tabular}

\begin{thebibliography}{11}
\bibitem{RedesdeComputadores}
Douglas E.Comer.
\textit{Redes de Computadores e Internet-6}
Bookman Editora, 2016.
\bibitem{RedesdeComp}
  Kurose, J. F., Ross, K. W.,
\textit{Redes de Computadores e a Internet.}
  Jornal - Uma nova,
  2006.
\bibitem{RedesComput}
Torres, Gabriel.
\textit{Redes de computadores}
Novaterra Editora e Distribuidora LTDA, 2015.
\end{thebibliography}

\end{document}